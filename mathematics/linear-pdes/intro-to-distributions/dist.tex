\documentclass[12pt, a4]{article}
\usepackage[utf8]{inputenc}
\usepackage{graphicx}
\usepackage{algorithm}
\usepackage{algpseudocode}
\usepackage{amsmath}
\usepackage{amsfonts}
\usepackage{hyperref}
\usepackage{mathtools}
\DeclarePairedDelimiter\ceil{\lceil}{\rceil}
\DeclarePairedDelimiter\floor{\lfloor}{\rfloor}

\hypersetup{
    colorlinks=true,
    linkcolor=blue,
    filecolor=magenta,
    urlcolor=cyan,
}

\usepackage[backend=bibtex]{biblatex}
\addbibresource{dist.bib}

\title{Linear Partial Differential Equations: Introduction to Distribution Theory}
\author{Srinath Kailasa \thanks{srinath.kailasa.18@ucl.ac.uk} \\ \small University College London}

\date{\today}



\newtheorem{theorem}{Theorem}[section]
\newtheorem{definition}{Definition}[section]
\newtheorem{notation}{Notation}[section]
\newtheorem{corollary}{Corollary}[theorem]
\newtheorem{lemma}[theorem]{Lemma}
\newtheorem{problem}[theorem]{Problem}

\DeclareMathOperator\supp{supp}
\DeclareMathOperator\reals{\mathbb{R}}
\DeclareMathOperator\complexes{\mathbb{C}}
\DeclareMathOperator\tfspace{C_0^\infty}
\DeclareMathOperator\tfspaceD{\mathcal{D}}
\DeclareMathOperator\dist{\mathcal{D'}}
\DeclareMathOperator\lone{L_{\text{loc}}^1}
\begin{document}

\maketitle

\section*{Abstract}

In this document I summarise some of the main applications of the theory of distributions for the solution of partial differential equations via the method of `fundamental solutions'. I begin by introducing the key concepts behind the idea of the method of distributions, with the goal of solving the Laplace and Heat equations using these ideas. Furthermore, this document also includes a `scratch pad' section, in which I provide further examples and proofs of the applications of these ideas.

\section{Distributions and Test Functions}

\subsection{Motivation}

Consider the wave equation in 1D,

\begin{equation}
    \frac{\partial^2u(x, t)}{\partial t^2} = \frac{\partial^2u(x, t)}{\partial x^2}
    \label{eq:wave_eq_1d}
\end{equation}

Any twice continuosly differentiable function $f \in C^2(\reals)$ of the form $u(x, t) = f(x-t)$ satisfies the wave equation. However, we face difficulties if the solution is admissable in a physical sense, but has mathematical problems such as being non-continuous, e.g. $u(x, t) = |x-t|$. This is the motivation behind seeking a new theory, a theory of distributions, which allow us to generalise notions of derivatives and solve problems with solutions that may not not be mathematicallly particlarly `nice' as they may be required by the physics of the problem. These generalisations are known as distributions.

\subsection{Defining Test Functions}

A few basic definitions will serve us well for the remainder of these notes.

\begin{definition}[Classes of Continuous Functions]
Let $\Omega^N$ be an open subset and let $m$ be a non-negative integer. The class $C^m(\Omega)$ consists of functions on $\Omega$ which have continuous derivatives of order less than or equal to $m$. Furthermore (1) $C^0(\Omega) = C(\Omega)$  is simply the class of all continuous functions on $\Omega$ and (2) $C^\infty (\Omega)$ is the class of functions with derivatives of all orders.
\label{def:c_m_functions}
\end{definition}

\begin{definition}[Support of Functions]
The support of a function $f : \Omega \rightarrow \mathbb{C}$ is the closure of the set $\{x \in \Omega | f(x) \neq 0\}$
\[ \supp f = \overline{\{x \in \Omega | f(x) \neq 0\}}\]
\label{def:support_of_functions}
\end{definition}

Examples in $\Omega = \reals$

\begin{itemize}
    \item $f(x) = 0$, then  $\supp f = \emptyset$
    \item $f(x) = x$, then $\supp f = \reals$
\end{itemize}

\begin{definition}
    $C_0^\infty(\reals^N)$ is the subset of $C^\infty(\reals^N)$ consisting of functions with \textbf{compact} support.
\end{definition}

\begin{definition}[Compact]
    Compact in this context means a \textbf{closed} and \textbf{bounded} set.
    \[ \Omega \subset \reals^N\]
    is bounded if $\exists R > 0$ such that
    \[\Omega \subset B_R = \{ x : |x| < R\}\]
    and $|x|$ can be understood as a length in $\reals^N$.
    \label{def:compact_function}
\end{definition}

Now we list some of the basic properties of the function space described by $\tfspace$, as this is the space from which we seek our coveted test functions. We can think of it as a linear vector space, with the usual properties. That is, for $\phi_1, \phi_2 \in \tfspace$ and $\alpha_1, \alpha_2 \in \complexes$,

\begin{eqnarray}
    \alpha_1\phi_1 + \alpha_2\phi_2  \in \tfspace
\end{eqnarray}

Before arriving at the final definition of test functions, we must also introduce the concept of the multi-index. This has a familiar whiff about if for those who are familiar with vector-calculus.

\begin{notation}[Multi-Index]
    A multi-index is a vector with $N$ components, \[ \alpha=(\alpha_1,...,\alpha_N) \] where each component is a non-negative integer. It has an \textbf{order}, described by the sum of all components $|\alpha| = \alpha_1+...+\alpha_N$. If $\beta$ is also a multi-index, then $\alpha+\beta$ is a component-wise sum.
\end{notation}

Using multi-indices we can denote the derivative of a function $f(x_1,x_2,...)$ with respect to a multi-index as,

\begin{eqnarray}
    \partial^\alpha f = \frac{\partial^{|\alpha|}}{\partial x_1^{\alpha_1}, \partial x_2^{\alpha_2}, ..., \partial x_N^{\alpha_N}}
\end{eqnarray}

Furthermore, for $\phi \in \tfspace$ any $\partial^\alpha \phi \in \tfspace$. If $a \in C^\infty$ i.e. a doesn't have compact support, $a\phi \in \tfspace$ - its product with a function with compact support will also have compact support. Finally changes of variables, will also leave us within $\tfspace$ i.e. for an arbitrary vector $b \in \reals^N$ and an $N \times N$ matrix $A$, $\phi(Ax+b) \in \tfspace$.

Now we have a promising vector space, with useful properties for it's members such as compactness and boundedness. There is however one more thing we need to consider, which is the idea of convergence within the vector space.

\begin{definition}
    A sequence $\{\phi_i\}^\infty_{i=1} \in \tfspace $ is said to converge to zero if
    \begin{enumerate}
        \item There exists a compact set $K \subset \reals^N$ such that $\supp \phi_j \subset K$ for all $i=1,2,....$
        \item For each multi-index $\alpha$ the derivative $\partial^\alpha \phi_j$ converges to zero uniformly as $j \rightarrow \infty$. That is, for all $\alpha$, $\underset{x \in \reals^N}{\supp} |\partial^\alpha \phi_j(x)|\rightarrow 0$
    \end{enumerate}

    \label{def:convergence_of_tf}
\end{definition}


The set $\tfspace$ with the above convergence properties is called the \textbf{space of test functions}, and denoted by $\tfspaceD$. We remark that the convergence property is often denoted with respect to $\tfspaceD$ in the literature as,

\begin{eqnarray}
    \phi_N \overset{\tfspaceD}{\rightarrow} \phi
\end{eqnarray}

which further implies,

\begin{eqnarray}
    \phi_N - \phi \overset{\tfspaceD}{\rightarrow} 0
\end{eqnarray}

\subsection{Defining Distributions}

Consider two abstract sets $A$ and $B$, the map between them $f : A \rightarrow B$ defines a \textbf{functional}. The map $f$ is considered to be continuous if $a_n \overset{A}{\rightarrow} a$ implies that $f(a_n) \overset{B}{\rightarrow} f(a)$. Strictly, this is `sequential continuity'.

\begin{definition}
    The set of distributions, $\dist$, is the set of all \textbf{linear} and \textbf{ (sequentially) continuous} functionals that map $f : \tfspaceD \rightarrow \complexes$
\label{def:distribution}
\end{definition}

We see that the distributions are defined by mapping from the space of test functions to (potentially complex) numbers. This concept will allow us to generalise notions of differentiability.

If $f$ is a functional, then

\begin{enumerate}
    \item For each $\phi \in \tfspaceD$, the functional associates a number denoted by the following Bra-Ket notation $\langle f, \phi \rangle \in \complexes$. This is also described as the \textbf{action} of the functional.
    \item The functional is linear such that for $\alpha_1, \alpha_2 \in \complexes$ and $\phi_1, \phi_2 \in \tfspaceD$, $\langle f, \alpha_1\phi_1, \alpha_2\phi_2 \rangle = \alpha_1\langle f, \phi_1 \rangle + \alpha_2\langle f, \phi_2 \rangle $
    \item A distribution $f$ is called (sequntially) continuous on $\tfspaceD$ if $\langle f, \phi_k \rangle \rightarrow \langle f, \phi \rangle$ where $\phi_k \overset{\tfspaceD}{\rightarrow} \phi$ as $k \rightarrow \infty$
\end{enumerate}

Let us illustrate the concept of functionals which satisfy the definition of distributions by going through some examples.

\begin{enumerate}
    \item \textbf{Continuous functions as distributions}. Let $f \in C(\reals^N)$ and define a functional with the mapping $\tfspaceD(\reals^N) \rightarrow \complexes$ as $\langle f, \phi \rangle = \int_{\reals^N} f(x) \phi(x) dx < +\infty$ which is valid for $\forall \phi \in \tfspaceD$. The integral is linear, by inspection, so it remains to show that it is also continuous in the sense of sequential continuity. Consider a sequence of test functions $\phi_n \in \tfspaceD$ such that $\phi_n \rightarrow \phi$ in $\tfspaceD$, without loss of generality let us assume that $\phi = 0$ so we have to show now that the action $\langle f, \phi_n \rangle \rightarrow 0$. From the definition of convergence in $\tfspace$ (\ref{def:convergence_of_tf}), we know that there exists a compact set $K$ for which $\supp \phi_n \subset K \> \> (\forall n)$. Then, \[ |\langle f, \phi_n \rangle | = \left |\int_{\reals^N}f \phi_n dx \right | = \left |\int_K f \phi_n dx \right | \leq \sup_{x \in \reals^N} |\phi_n| \int_K |f| dx\]. From (\ref{def:convergence_of_tf}) we see that $\sup_{x \in \reals^N} |\phi_n| \rightarrow 0$ and the integral over $|f|$ in $K$ is finite due to the compactness of $K$, therefore $\langle f, \phi_n \rangle \rightarrow 0$ and the functional is sequentiallly continuous. A remark about notation, we've identified above a function $f$ with a \textit{functional} which we also denote by $f \in \dist$.
    \item \textbf{Heaviside function}. It therefore follows from above that a function defines a distribution if the following statement holds, \[ \int_K |f| dx < + \infty, \> \> \forall K \subset \reals^N \]

    Consider the Heaviside function in one dimension, \[ H(x) = \begin{cases}
        1, x > 0 \\
        0, x \leq 0
    \end{cases}\]
    Then we can define the action of the distribution generated by the Heaviside function as, \[ \langle H, \phi \rangle = \int_{-\infty}^{\infty} H(x) \phi(x) dx = \int_{0}^{\infty} \phi(x) dx, \> \> \forall \phi \in \tfspaceD \]

    Therefore we see that it also identifiable as a distribution.
\end{enumerate}

Before proceeding, we need the concept of regular distributions,

\begin{definition}[Regular Distribution]
A distribution $f \in \dist$ is called regular if $\exists f \in \lone$ such that \[ \langle f, \phi \rangle = \int_{\reals^N} f(x) \phi(x) dx, \> \> \forall \phi \in \tfspaceD\]
    \label{def:regular_dist}
\end{definition}

Here $\lone$ defines a function space where each member is `locally integrable', i.e. has a finite integral on every compact subset of its domain of definition. More formally, let $\Omega$ be an open set in $\reals^N$ and $f : \Omega \rightarrow \complexes$ if the following integral holds,
\begin{equation}
    \int_K |f| dx < + \infty
    \label{eq:l_one_loc}
\end{equation}

for all subsets $K \subset \Omega$, then $f \in \lone$. The 1 refers to the power of the absolute value in the integral. We then notice that the conditions for sequential continuity are satisfied if a functional is chosen from $\lone$.

\vspace{5pt}

As a final example of this section, consider the Dirac $\delta$-function, defined by the following action,

\begin{eqnarray}
\langle \delta, \phi \rangle = \phi(0), \> \> \forall \phi \in \tfspaceD(\reals^N)
    \label{eq:dirac}
\end{eqnarray}

The functional is clearly in $\lone$, thefore satisfies sequential continuity, with linearity obvious by inspection. Therefore we see that it satisfies the properties required of a distribution.

\subsection{Basic Operations on Distributions}

Here, we state some basic facts about operations on distributions largely without proof.

\subsubsection{Multiplication by a $C^\infty$ function}
For example, If $a \in C^\infty(\reals^N)$ and $f \in C(\reals^N)$ then $f \in \dist$ then,

\begin{equation}
    \langle af, \phi\rangle = \int_{\reals^N}(a(x)f(x))\phi(x)dx = \int_{\reals^N}f(x) (a(x)\phi(x))dx = \langle f, a\phi \rangle
\end{equation}

Implying that $a\phi \in \tfspace$. This identity can be extended to all distributions.

\begin{definition}
    If $a \in C^\infty$ and $f \in \dist$ then, $af \in \dist$ defined by \[
        \langle af, \phi \rangle = \langle f, a\phi \rangle, \>\> \forall \phi \in \tfspaceD\]
\end{definition}

One can check that the above does define a distribution through an example. Consider $f = \delta \in \dist$ and $a \in \tfspace$, then

\begin{eqnarray}
    \langle a\delta, \phi \rangle = \langle \delta, a \phi \rangle = a(0)\phi(0) = a(0) \langle \delta, \phi \rangle \> \> \forall \phi \in \tfspaceD
\end{eqnarray}

This implies that $a\delta = a(0) \delta$ and does define a distribution.

\subsubsection{Differentiation}

Differentiation for distributions uses a form of Green's identity. Consider the one dimensional case, $f \in C(\reals)$ and $\phi \in \tfspaceD(\reals)$ and has compact support, then

\begin{equation}
    \int_{-\infty}^{\infty} f'(x) \phi(x) dx = \int_a^b f'(x)\phi(x)dx = \left [ f \phi \right ]_a^b - \int_a^b f\phi' dx = -\int_a^b f\phi' dx
\end{equation}

The boundary term vanishes due to the compact support of $\phi$. Generalising the above identity for higher order derivatives, we see that,

\begin{equation}
    \int_{-\infty}^{\infty} \frac{d^m}{dx^m} f(x) \phi(x) dx = (-1)^m \int_{-infty}^{\infty} f(x) \frac{d^m}{dx^m}\phi(x) dx
\end{equation}

In the language of actions this goes to,

\begin{eqnarray}
    \langle \frac{d^m}{dx^m}f, \phi \rangle = \langle f, \frac{d^m}{dx^m} \phi \rangle
\end{eqnarray}

Furthermore, we state that this differentiated object defines a distribution too.

\begin{definition}
    Let $f \in \dist(\reals^N)$. Then for any multi-index $\alpha$ the distribution $\partial^\alpha f$ is defined to be, \[ \langle \partial^\alpha f, \phi \rangle = (-1)^{|\alpha|} \langle f, \partial^\alpha \phi \rangle, \> \> \forall \phi \in \tfspaceD(\reals^N) \]
    \label{def:differentiate_distribution}
\end{definition}

Let's consider some simple corollary examples,

\begin{enumerate}
    \item Consider a one dimensional problem in $\reals$ where $\delta \in \dist$. What is $\frac{d}{dx}\delta$? From the above definition, \[
        \langle \frac{d}{dx} \delta, \phi\rangle = - \langle \delta, \frac{d}{dx}\phi \rangle = -\phi'(0)\]
    \item Consider again a one dimension problem with the Heaviside function $H \in \dist$. What is $\frac{d}{dx}H$? Using the same procedure, \[
        \langle \frac{d}{dx}H, \phi \rangle = -\langle H, \frac{d}{dx} \phi \rangle = \int_0^\infty \frac{d\phi}{dx}dx = -[\phi]_0^\infty = \phi(0) = \langle \delta, \phi \rangle, \> \> \forall \phi \in \tfspaceD \]. This implies that $\frac{dH}{dx} = \delta$, in a distributional sense.
\end{enumerate}

\subsubsection{Change of Variables ($x = Ay + b$)}

Finally let's consider a linear transformation, corresponding to a change of variables. Here $A$ is a non-singular $N \times N$ matrix and $b \in \reals^N$. If $f \in C^\infty(\reals^N)$ and $\phi \in \tfspaceD(\reals^N)$, then

\begin{equation}
    \int_{\reals^N}f(Ay+b)\phi(y)dy = \int_{\reals^N}f(x)\phi(A^{-1}(x-b))\frac{1}{\text{det}(A)}dx
\end{equation}

This also defines a distribution, giving us our final definition in this section,

\begin{definition}
    Let $f \in \dist(\reals^N)$, with $A$ and $b$ as above, then $f(Ay+b) \in \dist$ is defined by \[\langle f(Ay+b), \phi(y)\rangle = \langle f(x), \phi(A^{-1}(x-b))\frac{1}{|A|}\rangle, \> \> \forall \phi \in \tfspaceD(\reals^N)\]
\end{definition}

\subsection{The Support of Distributions}

\subsection{The Convergence of Distributions}

\subsection{Equivalent Definition of Distributions}


\section{Fundamental Solutions}


\section{Scratch Pad}

\begin{problem}

\end{problem}

\end{document}
