\documentclass[12pt, a4, twoside]{article}
\usepackage[utf8]{inputenc}
\usepackage{graphicx}
\usepackage{algorithm}
\usepackage{algpseudocode}
\usepackage{amsmath}
\usepackage{amsfonts}
\usepackage{hyperref}
\usepackage{mathtools}
\DeclarePairedDelimiter\ceil{\lceil}{\rceil}
\DeclarePairedDelimiter\floor{\lfloor}{\rfloor}

\hypersetup{
    colorlinks=true,
    linkcolor=blue,
    filecolor=magenta,
    urlcolor=cyan,
}

\usepackage[backend=bibtex]{biblatex}
\addbibresource{boundary-integral-equations.bib}

\title{Integral Equation Methods for Scattering}
\author{Srinath Kailasa \thanks{srinath.kailasa.18@ucl.ac.uk} \\ \small University College London, \small Flatiron Institute}

\date{\today}

\begin{document}

\maketitle

\section*{Abstract}
Integral equations are an invaluable tool in computational science for formulating problems which compute over unbounded domains, such as acoustic or electromagnetic scattering, as they allow for a reduction in problem dimension. The focus of this document is to summarise the main aspects of integral equations as they relate to scattering problems. The focus is on clarity of presentation rather than neatness of proof! Resultantly these notes form a Physicist's (`baby mathematician') attempt to rationalise and clarify the beautiful theorems from this fascinating field. These notes were written up during my visit to the Flatiron Institute in New York City in the Summer of 2022.
\end{document}
