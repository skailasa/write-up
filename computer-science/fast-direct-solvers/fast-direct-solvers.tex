\documentclass[12pt, a4, twoside]{article}
\usepackage[utf8]{inputenc}
\usepackage{graphicx}
\usepackage{algorithm}
\usepackage{algpseudocode}
\usepackage{amsmath}
\usepackage{amsfonts}
\usepackage{hyperref}
\usepackage{mathtools}
\DeclarePairedDelimiter\ceil{\lceil}{\rceil}
\DeclarePairedDelimiter\floor{\lfloor}{\rfloor}

\hypersetup{
    colorlinks=true,
    linkcolor=blue,
    filecolor=magenta,
    urlcolor=cyan,
}

\usepackage[backend=bibtex]{biblatex}
\addbibresource{fast-direct-solvers.bib}

\title{Fast Direct Solvers for the Solution of Integral Equations}
\author{Srinath Kailasa \thanks{srinath.kailasa.18@ucl.ac.uk} \\ \small University College London}

\date{\today}

\begin{document}

\maketitle

\section*{Abstract}
So called `fast direct solvers' offer an $O(N)$ alternative to iterative methods ($O(n_{iter} \cdot n)$) for the solution of integral equations, and therefore are a rapidly developing field of research. In this document, I summarise the recent research in this direction in the context of computing the solution of acoustic and electromagnetic scattering problems which have been formulated as integral equations. \footnote{These notes were written up during my visit to the Flatiron Institute in New York City in the Summer of 2022. A wonderful experience for which I am extremely grateful. The visit gave me both the impetus and the time to study some truly interesting and beautiful concepts.}.

\newpage

\tableofcontents

\section{A Hierarchical Matrix Zoo}

The matrices that we're concerned with are in some sense `data-sparse', in that the off-diagonal elements are in some way `small', and compressible without a great deal of information loss.

Figures (\ref{fig:hierarchical_matrices}) and (\ref{fig:hierarchical_matrices_2}) from Ambikasaran \cite{Ambikasaran2013}, summarizes the main types of matrices that arise when discretising integral equations and their relationship to each other. In the context of hierarchical matrices and fast direct solvers, \textit{\textbf{Admissibility}} refers to whether neighbour boxes are considered low-rank (weak), or whether only non-adjacent children of the parent box's neighbours are considered low-rank (strong). \textit{\textbf{Nested Basis}}, are related to the ability to form the basis for a given box using that of its children.


\begin{figure}[!h]
    \centering
    \includegraphics[width=0.6\textwidth]{euler.png}
    \caption{Taxonomy of hierarchical matrices.}
    \label{fig:hierarchical_matrices}
\end{figure}


\begin{figure}[!h]
    \centering
    \includegraphics[width=0.6\textwidth]{table.png}
    \caption{How to classify hierarchical matrices.}
    \label{fig:hierarchical_matrices_2}
\end{figure}

\section{Strong Recursive Skeletonization: RS-S}



\section{Proxy Compression}

Proxy compression is necessary in order to achieve the linear complexity bound of the fast-direct solver powered by RS-S. The idea rests on the principle of representing the far-field particles of a given box $B$, which may contain $O(N)$ particles, with a set of `proxy points' contained on a proxy surface that encloses $B$. This surface is often chosen to be a sphere. By choosing $O(1)$ proxy points, without getting into the details yet of how exactly they are sampled, we are able to obtain the linear complexity we desire.

\begin{figure}[!h]
    \centering
    \includegraphics[width=0.6\textwidth]{outgoing_proxy.jpg}
    \caption{Considering the outgoing problem due to charge contained on $\Gamma \cap B$ evaluated in the far-field of $B$.}
    \label{fig:outgoing_proxy}
\end{figure}


For a given box $B$, a proxy surface $D$ and its boundary $\gamma$ are chosen such that $B \subset D$. The far-field points of $B$, $\mathcal{F}$ is partitioned such that $\mathcal{F} = \mathcal{Q} \cup \mathcal{P}$, where $\mathcal{Q}$ contains $O(1)$ points. $\Gamma$ is the boundary of the entire scatterer, and $\tau = \Gamma \cap B$ is the portion of the scatterer boundary contained in $B$. The situation is sketched in figure (\ref{fig:outgoing_proxy}) in 2D.

We can choose to represent our solution due to the charge in $B$ in $\mathcal{F}$ however we wish. However, our choice will lead to different matrices that we must compress. 

Generally, we'll end up with a solution matrix of the form $A_{\mathcal{F}B}$ that maps between $B$ and points in its far-field that can be split up as,

\begin{flalign}
    v_{\mathcal{F}} &= A_{\mathcal{F}B} \psi_B \\
    &= B_{\mathcal{F}\gamma} C_{\gamma B} \psi_B
\end{flalign}

the subscripts indicate the domains these operators map between. We desire a split like this, as the far field interaction of $B$ can be compressed into something involving $C_{\gamma B}$.

To see this, consider the fact that $A_{\mathcal{F}B}$ can be written as,

\begin{flalign}
    \label{eq:decomposition}
    A_{\mathcal{F}B} = \begin{bmatrix}
        A_{\mathcal{Q}B}\\ A_{\mathcal{P}B}
        \end{bmatrix} = \begin{bmatrix}
        I & 0\\ 0 & B_{\mathcal{P}\gamma}
        \end{bmatrix} \begin{bmatrix}
        A_{\mathcal{Q}B}\\ C_{\gamma B}
        \end{bmatrix}
\end{flalign}

Our RS-S algorithm relies on a compression of this matrix, however a direct compression of $A_{\mathcal{F} B}$ is too expensive, as there are typically $O(N)$ points in $\mathcal{F}$ for a box $B$, therefore assembly of this matrix for all boxes will result in an algorithm of $O(N^2)$ complexity. However if we can find a decomposition like above, we can apply an interpolative decomposition to the right column in (\ref{eq:decomposition}) which has dimensions $O(1) \times O(n_\gamma)$ by construction where $n_\gamma$ is the number of proxy points. To prove that this allows us to reconstruct the full matrix after compression. Consider an ID that gives us,

\begin{flalign}
    \begin{bmatrix}
        A_{\mathcal{Q}B}\\ C_{\gamma B}
        \end{bmatrix} = \begin{bmatrix}
            A_{\mathcal{Q}S}\\ C_{\gamma S}
        \end{bmatrix} \begin{bmatrix}T_{SR}  & 1 \end{bmatrix}
\end{flalign}

Where $S$ and $R$ are the skeleton and redundant points respectively. Plugging back into our expression (\ref{eq:decomposition}),

\begin{flalign}
    \label{eq:compressed}
    A_{\mathcal{F}B} &= 
        \begin{bmatrix}
            I & 0\\ 0 & B_{\mathcal{P}\gamma} 
        \end{bmatrix}    
        \begin{bmatrix}
             A_{\mathcal{Q}S}\\ C_{\gamma S}
        \end{bmatrix}
        \begin{bmatrix}T_{SR}  & 1 \end{bmatrix} \\
        &=\begin{bmatrix}
            A_{\mathcal{Q}S}\\ B_{\mathcal{P} \gamma}C_{\gamma S} 
       \end{bmatrix} \begin{bmatrix}T_{SR}  & 1 \end{bmatrix} \\
       &=  A_{\mathcal{F}S} \begin{bmatrix}T_{SR}  & 1 \end{bmatrix}
\end{flalign}

Therefore, we see that we can get away with a cheap ID to reconstruct the far-field operator, involving the proxy points rather than the full far field of $B$.

\subsection{$\mathcal{T}$}

\subsubsection{Outgoing Skeletonization}

A double-layer potential, due to some unknown density $\psi$, supported on $\tau$,

\begin{flalign}
    v(x) = \int_{\Gamma \cap B} \frac{\partial \Phi(x, y)}{\partial n(y)} \psi(y) ds(y) := \mathcal{D}\psi, \> \> x \in \mathbb{R}^m \setminus \tau 
\end{flalign}

solves the Helmholtz equation everywhere it's valid. Here, $\Phi(x, y)$ is the fundamental solution of the Helmholtz equation. However, its normal derivative evaluated at the target points, which we'll need for deriving boundary integral equations for Maxwell problems, does not,

\begin{flalign}
    \frac{\partial v}{\partial n(x)} = \frac{\partial}{\partial n(x)} \int_{\Gamma \cap B} \frac{\partial \Phi(x, y)}{\partial n(y)} \psi(y)ds(y) := \mathcal{T}\psi, \> \> x \in \Gamma \cap \mathcal{F}
\end{flalign}

it's only valid at far-field points, $\Gamma \cap \mathcal{F}$. However, we can separate out the normal part of the derivative,

\begin{flalign}
    \frac{\partial v}{\partial n(x)} = n(x) \cdot \nabla_x \int_{\Gamma \cap B} \frac{\partial \Phi(x, y)}{\partial n(y)} \psi(y)ds(y) := n \cdot w
\end{flalign}

The function

\begin{flalign}
    w(x) = \nabla_x \int_{\Gamma \cap B} \frac{\partial \Phi(x, y)}{\partial n(y)} \psi(y)ds(y) := \nabla_x \mathcal{D}\psi
\end{flalign}

Does satisfy our PDE, everywhere, and we'll exploit this fact in a moment. As an aside, we can see that this is true by considering a double layer potential $v$ that is smooth enough to admit,

\begin{flalign}
    (\Delta + k^2)w =(\Delta + k^2)\nabla_x v =\nabla_x (\Delta + k^2) v = 0
\end{flalign}

where the last equality follows as $v$ satisfies the Helmholtz equation. Therefore $w$ is a solution of the Helmholtz equation. Note that $w$ has three components.

In order to find our $C_{\gamma B}$ with this representation, we need to set up an `associated boundary value problem' for each component of $w$. The choice of boundary value problem we choose is free, as we only rely on the existence of its solution.

Consider an associated boundary value problem for just a single component of $\tilde{w}$ that satisfies,

\begin{flalign}
    &(\Delta + k^2)\tilde{w} = 0, \> \> x \in \mathbb{R}^m \setminus D \\
    &\tilde{w} = w_1(x) \\
    &\text{A radiation condition at } \infty
\end{flalign}

A combined field representation might be nice, as we know it has good properties,

\begin{flalign}
    \tilde{w} = (\mathcal{D} - ik \mathcal{S})_{\mathcal{F}\gamma} \mu
\end{flalign}

where $\mu$ is some unknown density supported on the proxy surface $\gamma$. Forming the boundary integral equation, and plugging back into the representation for $\tilde{w}$,

\begin{flalign}
    \tilde{w} &=  (\mathcal{D} - ik \mathcal{S})_{\mathcal{F} \gamma}(\frac{1}{2}\mathcal{I} + \mathcal{D} - ik \mathcal{S})_{\gamma \gamma}^{-1}w_1 \\
    &= (\mathcal{D} - ik \mathcal{S})_{\mathcal{F} \gamma}(\frac{1}{2}\mathcal{I} + \mathcal{D} - ik \mathcal{S})_{\gamma \gamma}^{-1} \nabla_1 \mathcal{D}_{\gamma B} \psi_\gamma \\ 
    &\equiv B_{\mathcal{F}\gamma} C_{\gamma B} \psi_\gamma
\end{flalign}

where we identify,

\begin{flalign}
    C_{\gamma B} = \nabla_1 \mathcal{D}_{\gamma B} 
\end{flalign}

This is the matrix we will attempt to compress. Similar analysis follows for the other two components of $w(x)$. Meaning that we end up having to compress $[\nabla_1 \mathcal{D}_{\gamma B} , \nabla_2 \mathcal{D}_{\gamma B} , \nabla_3 \mathcal{D}_{\gamma B}]$ for the outgoing problem.

We see that $B_{\mathcal{F} \gamma }$ is never explicitly formed, we just require its existence. When we calculate an approximation of $A_{\mathcal{F}B}$ using (\ref{eq:compressed}), we only need to know the ID of the $C_{\gamma B}$.

\subsubsection{Incoming Skeletonization}

For the incoming skeletonization, were again we're considering the same representation with a hypersingular operator, we observe that we're just looking for,

\begin{flalign}
    \left [\frac{\partial v}{\partial n(x)} \right ]_{\mathcal{F} B}^T
\end{flalign}

with the formation of an associated boundary integral equation taking place in much the same way as for the outgoing problem. However, the double layer operator is self-adjoint, therefore it leads to the same expressions for $C_{\gamma B}$.

\subsection{$\mathcal{K}'$}

\subsubsection{Outgoing Skeletonization}

If we choose to represent our potential with a single-layer potential,

\begin{flalign}
    u(x) = \int_{\Gamma \cap B} \Phi(x, y) \phi(y) ds(y) := \mathcal{S}\phi, \> \> x \in \mathbb{R}^m \setminus D
\end{flalign}

and seek a boundary integral equation in terms of its normal derivative at the targets,

\begin{flalign}
    w(x) = \int_{\Gamma \cap B} \frac{\partial \Phi(x, y)}{\partial n(x)} \phi(y) ds(y) := \mathcal{K}'\phi, \> \> x \in \Gamma \cap \mathcal{F}
\end{flalign}

We observe the same problem as in the $\mathcal{T}$ case, where this expression is not a general solution of our PDE. We can similarly separate out the normal component and write,

\begin{flalign}
    \tilde{w}(x) := \int_{\Gamma \cap B} \nabla_x \Phi(x, y) \phi(y) ds(y), \> \> x \in \mathbb{R}^m \setminus D
\end{flalign}

Using the previous analysis for $\mathcal{T}$, we immediately recognise that the components we must compress are $C_{\gamma B} = \nabla_1 \mathcal{S}_{\gamma B}$, giving us $[\nabla_1 \mathcal{S}_{\gamma B}, \nabla_2 \mathcal{S}_{\gamma B}, \nabla_3 \mathcal{S}_{\gamma B}]$ to compress in total for the outgoing problem.

\subsubsection{Incoming Skeletonization}

Noticing that,

\begin{flalign}
    \left [\frac{\partial u}{\partial n(x)} \right ]_{\mathcal{F} B}^T = \int_{\Gamma \cap B} \frac{\partial \Phi(x, y)}{\partial n(y)} \phi(y) ds(y) = \mathcal{D}_{\gamma B} \phi
\end{flalign}

already satisfies our PDE without any further work, we can save a lot of work, and simply use it as our Dirichlet data in the associated boundary value problem. The matrix to compress being $C_{\gamma B} = \mathcal{D}_{\gamma B}$.


\end{document}


