\documentclass[12pt, a4, twoside]{article}
\usepackage[utf8]{inputenc}
\usepackage{graphicx}
\usepackage{algorithm}
\usepackage{algpseudocode}
\usepackage{amsmath}
\usepackage{amsfonts}
\usepackage{hyperref}
\usepackage{mathtools}
\DeclarePairedDelimiter\ceil{\lceil}{\rceil}
\DeclarePairedDelimiter\floor{\lfloor}{\rfloor}

\hypersetup{
    colorlinks=true,
    linkcolor=blue,
    filecolor=magenta,
    urlcolor=cyan,
}

\usepackage[backend=bibtex]{biblatex}
\addbibresource{fast-direct-solvers.bib}

\title{Fast Direct Solvers for the Solution of Integral Equations}
\author{Srinath Kailasa \thanks{srinath.kailasa.18@ucl.ac.uk} \\ \small University College London}

\date{\today}

\begin{document}

\maketitle

\section*{Abstract}
So called `fast direct solvers' offer an $O(N)$ alternative to iterative methods ($O(n_{iter} \cdot n)$) for the solution of integral equations, and therefore are a rapidly developing field of research. In this document, I summarise the recent research in this direction in the context of computing the solution of acoustic and electromagnetic scattering problems \footnote{These notes were written up during my visit to the Flatiron Institute in New York City in the Summer of 2022. A wonderful experience for which I am extremely grateful. The visit gave me both the impetus and the time to study some truly interesting and beautiful concepts.}.

\newpage

\tableofcontents


\end{document}


