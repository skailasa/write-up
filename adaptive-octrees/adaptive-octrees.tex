\documentclass[12pt, a4, twoside]{article}
\usepackage[utf8]{inputenc}
\usepackage{graphicx}
\usepackage{amsmath}
\usepackage{amsfonts}
\usepackage{hyperref}
\hypersetup{
    colorlinks=true,
    linkcolor=blue,
    filecolor=magenta,
    urlcolor=cyan,
}

\usepackage[backend=bibtex]{biblatex}
\addbibresource{adaptive-octrees.bib}

\title{Adaptive Octrees, Algorithms and Implementations}
\author{Srinath Kailasa \thanks{SK is funded by an EPSRC DTP at UCL}}

\date{\today}

\begin{document}

\maketitle

\section*{Abstract}

Octrees find a wide variety of applications across computational science as they
allow for the spatial decomposition of a three dimensional domain. The adaptivity of
an octree refers to the potential to have neighbouring tree nodes of non-uniform
size at a given level of the octree. This allows for a better description of
non-uniformly distributed data, allowing the octree to `adapt' to the data. In this
document we describe the main theoretical underpinnings of adaptive octrees, with
special reference to their application in Fast Multipole Methods. We move on to
a description of major sequential and parallel implementations of adaptive octrees,
and describe the strategy being undertaken by the new Python/SYCL implementation
being developed as a part of the Excalibur exascale software collaboration \cite{excalibur},
\href{https://github.com/Excalibur-SLE/AdaptOctree}{AdaptOctree}.

\cite{Sundar:2008:SIAM}

\section*{Motivation}


\section*{Algorithms}


\section*{AdaptOctree Software}


\printbibliography

\end{document}